\documentclass[]{article}

\usepackage{listings}
\usepackage{amsmath}
\usepackage{amssymb}
\usepackage{graphicx}
\usepackage{setspace}
\lstset{language=Python}

\title{Detecting Racism in Tweets : Progress Report}
\author{Fiete Botschen, Sean Noran}

\begin{document}

\maketitle

\onehalfspacing

\section{Goal Summary}
Our goal is to learn a supervised model that can detect racism in Tweets.

\section{Labeled Data Collection}

We are currently using a subset of a substantially large Twitter dataset. The data we used span over 20 days in January 2011 and 20 days in March 2015 and totals approximately 70 GB. This equates to nearly 30 million Tweets, including not only the Tweet itself but also various metadata pertaining to the Tweet, such as geolocation, Twitter account, number of times the Tweet was favorited, etc. The average Tweet contains about 30 characters and may contain links to images, news articles and other web content.

In addition to this available database of Tweets, we used a Twitter crawler to acquire a dataset of 6309 Tweets posted by a Twitter use who posts racist jokes and another Twitter user that retweets racist Tweets.

\subsection{Pre-Processing}
The first two datasets contain many Tweets in languages other than English. In order to isolate only English language Tweets, we filter our data by the language tag attached to each Tweet.

\subsection{Data Filtering}

The raw data is largely imbalanced; that is, racist Tweets occur infrequently. Thus generating an adequately large set of balanced data would be rather time-consuming. We therefore decide to filter the Tweets by words related to race. The list of words is not included in this report due to its explicit content.

For the dataset over January 2011, we filtered by explicit racial derogatory terms. In addition, we filtered the data by colors such as 'white' and 'black' and groups of people such as 'Mexicans' and 'Africans'. While many racist Tweets do include these terms, we found that most of the Tweets including these words were not considered racist, whereas the derogatory terms alone were better at isolating racism. This produced just over 70,000 Tweets.

For the dataset over March 2015, we filtered only by racial derogatory terms. This resulted in only around 10,000 Tweets to label.

\subsection{Data Labeling}

For the first dataset, we labeled 4000 Tweets manually ourselves and found 167 (4.175\%) of them to be racist.

For the second dataset, we labeled 4932 Tweets and found 426 (8.64\%) to be racist.

Because the labeling process was cumbersome for so few Tweets, we additionally decided to acquire data from a Twitter user who posts primarily racist content. Of the 2000 Tweets from this dataset we have thus far labeled, 54.17\% of these Tweets are racist, far exceeding the proportion of our previous labeled datasets. Figure \ref{fig:word_count} shows the frequency of the most common words found in these Tweets, excluding the 100 most common English words.

\begin{figure}[h]
\label{fig:word_count}
\caption{Frequency of most common words in 6308 Tweets involving racist Twitter users (excluding 100 most common English words)}
\includegraphics[width=13cm]{word_count}
\centering
\end{figure}

\subsection{Subjective Labeling}

One of the major questions involved in the data collection process is: What is racism? 

The U.S. Commission on Civil Rights defines racism as "any action or attitude, conscious or unconscious, that subordinates an individual or group based on skin colour or race. It can be enacted individually or institutionally"\footnote[1]{Racism In America and How to Combat It, U.S. Commission on Civil Rights, 1970}.

However, there is vast disagreement regarding how to define "race" and "subordinate". And indeed many alternative definitions of racism exist among various academic fields such as Sociology, Psychology, Political Science and Biology.

For instance, we disagreed on the content of the following Tweet:
\\\\
	RT @DeadlySushi: @YesYoureRacist Instead of complaining about police fix youre neighborhoods first and stop doing crimes! Point the finger …
\\\\
After seeing many Tweets about the same context - the shooting of African American Michael Brown in Ferguson, Missouri - one of us felt that this Tweet was racist, while the other did not.

Because many people may disagree on which Tweets are racist, it would be better to define a confidence score as the percentage of people who said a given Tweet is racist. By this method, we would expect blatant racism to receive a confidence score close to 1, clearly non-racist Tweets to receive a score close to 0 and ambiguous Tweets to fall somewhere in between, close to 0.5.

However, it is also important to note that racism may be defined differently by different groups of people. In particular, an ethnic minority may find a certain Tweet to be racist, whereas a Caucasian may not see the racism in it.

Thus, if we are to collect data over a larger population of people, it would be important to account for many demographics. Additionally, we would want participants who themselves are not racist.

\section{Evaluation}

\subsection{Bag-of-Words Classification Results}

We evaluated three classifiers over binary bag-of-words features. Each model is trained on 4000 Tweets and evaluated on 2000 Tweets. In both cases, half of the Tweets are from the third dataset from the Twitter crawl and the remaining half are from random Tweets on Twitter. For each Tweet, a binary feature is associated with each word in our vocabulary indicating whether the word appears in that Tweet. The top 100 most common English words are excluded.

Note that these Tweets were not labeled individually as racist or not; instead, all Tweets from the first half were given the label 'racist' and all tweets from the second half were given the label 'non-racist'. We are currently finishing the labeling process and will re-evaluate our models thereafter. This is important, because we found that only around 54\% of the Tweets in the first half were actually racist content.

The results for each of three classifiers are shown below:

\begin{enumerate}
\item Naive Bayes: 72.889081832\%
\item Support Vector Machine: 72.9976123291\%
\item Random Forest: 73.258085522\%
\end{enumerate}


We also found that if we vary the proportion of racist Tweets in our test dataset, the accuracy changes as follows:

\begin{enumerate}
\item 200 non-racist, 2000 racist: ~55\%
\item 2000 non-racist, 2000 racist: ~73\%
\item 20000 non-racist, 2000 racist: ~88\%
\end{enumerate}

This indicates that the models that we use tend to classify Tweets as non-racist more frequently than racist, despite the fact that our training data was balanced. This suggests that we need to improve feature engineering in order to achieve better accuracy.

\section{Timeline}

Our next steps involve

\begin{enumerate}
	\item By 11/20: Reading additional existing research on similar work
	\item By 11/27: Feature Engineering
	\item By 12/01: Evaluating several classifiers
	\item By 12/04: Comparing our results to those of existing work
	\item By 12/08: Completed slides for presentation
	\item By 12/18: Finish project report
	\item By 12/24: Enjoy Christmas cookies
\end{enumerate} 

\end{document}